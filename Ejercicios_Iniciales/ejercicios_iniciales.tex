% 	TEMPLATE DE RELACIÓN DE EJERCICIOS
%
% Creador: Juamagdev

% ------------------------------------------------------------------
% ---------------------- Package imports ---------------------------
% ------------------------------------------------------------------
\documentclass[11pt, a4paper]{exam}
\usepackage[utf8]{inputenc}		% UTF-8
\usepackage[spanish]{babel}		% Idioma
\usepackage{graphicx}			% Para importar gráficos
\usepackage{amsthm, amsfonts, amssymb, amssymb} % Todos los paquetes AMS
\usepackage{mathtools}          % Arregla bugs de AMS
\usepackage{hyperref}			% Para \href{URL}{text}
\usepackage{enumitem}			% Para enumerar
\usepackage{color} 				% Para definir colores nuevos
\usepackage{lastpage}			% Para \pageref{LastPage}
\usepackage{booktabs}					% Tablas profesionales
\usepackage{parskip}					% Espacio de párrafos
\usepackage[sharp]{easylist}			% Para litas
\usepackage[expansion=false]{microtype} % Soluciona bugs de tipografias
\usepackage[margin=2.25cm, includehead, includefoot]{geometry}
\usepackage{cancel}
\usepackage{pgfplots}
\usepackage{listings}



% ------------------------------------------------------------------
% ---------------------- Constantes --------------------------------
% ------------------------------------------------------------------
% Las constantes mas importantes
\newcommand{\mytitle}{Ejercicios iniciales}
\newcommand{\mysubject}{Algoritmia y Complejidad}
\newcommand{\mydate}{\today}
\newcommand{\myauthor}{Grupo 04}

% Redefinir para cada caso
\newcommand{\myrhead}{Universidad de Málaga}
\newcommand{\mypagename}{Página}
\newcommand{\mycreated}{Creado por: }
\newcommand{\mysolname}{Solución}
\renewcommand{\solutiontitle}{\noindent\textbf{\mysolname.}\hspace{0.75em}}
\pointpoints{point}{points}

\providecommand{\abs}[1]{\lvert#1\rvert}

% ------------------------------------------------------------------
% ---------------------- Ajustes -----------------------------------
% ------------------------------------------------------------------
% \shadedsolutions
\printanswers % Alternative: \noprintanswers, \printanswers
%\rhead{{\scshape {\footnotesize  \myrhead}}}
%\cfoot{\mypagename \enspace \thepage}
\definecolor{SolutionColor}{rgb}{0.8,0.9,1} % light blue

% Cabeceras y pie de página
\runningfootrule
\firstpagefootrule
\firstpagefooter{\mysubject}{}{\mypagename\ \thepage\ de \pageref*{LastPage}}
\runningfooter{\mysubject}{}{\mypagename\ \thepage\ de \pageref*{LastPage}}
\firstpageheadrule
\firstpageheader{\mydate}{\mytitle}{\pageref*{LastPage} páginas en total}

% Answer command for double lines
\def\answer#1{\underline{\underline{#1}}}

% ------------------------------------------------------------------
% ---------------------- Documento ---------------------------------
% ------------------------------------------------------------------
\begin{document}
\pagestyle{headandfoot}
\noindent {\scshape \Large  \mytitle
    \ifprintanswers
        \enspace (\mysolname	 )
    \fi
} \\
\noindent {\mycreated \enspace  \myauthor} \vspace{1em}
\hrule \hrule
\vspace{5mm}

\newcommand*{\codesetfont}{\fontfamily{pcr}\selectfont}
\newcommand{\X}[1]{\text{\codesetfont X#1}}
\newcommand{\WHILE}{\text{WHILE}}
\newcommand{\dokw}{\text{ \codesetfont do }}
\newcommand{\od}{\text{ \codesetfont od}}
\newcommand{\while}{\text{\codesetfont while }}


% ------------------------------------------------------------------
% ---------------------- Contenido ---------------------------------
% ------------------------------------------------------------------

\begin{questions}
    \addpoints
    \question {\bfseries Ejercicio 4}
    
    \begin{parts}
        \part Abstracción para obtener propiedades genéricas; por encima de los detalles.
        \begin{solution}
            \\
            La abstracción es el proceso de generalizar al reducir el contenido de información de un concepto o un fenómeno observable. 
            Consiste en captar las características esenciales de un objeto, así como su comportamiento, ignorando otras propiedades del objeto en cuestión.
            La abstracción es una habilidad esencial para la construcción de modelos y la descomposición de problemas.
            \\
            \\
            La abstracción sirve para la formación del conocimiento humano, ya que permite crear conceptos abstractos a partir de la observación de la realidad.
            También sirve para aspectos informaticos, como la programación orientada a objetos, ya que permite separar el comportamiento específico de un objeto y enfocarse en su visión externa.
            La abstracción facilita el análisis y la comprensión de los fenómenos, sistemas o procesos.
            \\
            \\
            Algunos ejemplos de abstracción son:
            \begin{enumerate}
                \item Representar objetos cotidianos con palabras o imágenes.
                \item Elaborar definiciones de conceptos como la democracia o la justicia.
                \item Diagnosticar una enfermedad a partir de los síntomas de un paciente.
                \item Identificar el problema de un vehículo a partir de los sonidos o las señales que emite.
                \item Clasificar animales según sus características comunes.
                
            \end{enumerate}

        \end{solution}

        \part Lenguaje WHILE
            
        \begin{solution}
            \\
            El lenguaje WHILE es un lenguaje de programación simple construido a partir de asignaciones
            , composición secuencial, condicionales y sentencias while. 
            Es un lenguaje imperativo simple con variables enteras y expresiones enteras y booleanas. 
            Para ramificación tiene una declaración if-else y para bucle tiene una declaración while.
            \\
            \\
            El lenguaje WHILE es Turing computable, lo que significa que puede resolver cualquier problema que pueda 
            ser resuelto por una máquina de Turing.
            En teoría, esto significa que el lenguaje WHILE puede resolver cualquier problema computable. Sin embargo, en la práctica, algunos problemas pueden ser muy difíciles de resolver con el lenguaje WHILE debido a limitaciones de tiempo y espacio. 
            \\
            \\
            La definición formal del lenguaje WHILE es la siguiente:
            \\
            \\
            Dados los alfabetos:\\
            $\Sigma_w = \{\X{}, \coloneqq, 1, +, -, \while, \neq, 0, \dokw, \od, ;\}$\\
            $\Sigma_d = \{0,1,2,3,4;5,6,7,8,9\}$\\

            % \begin{tabular}{ l l l l l}
                Sea $G = (N, T, P, <\text{code}>)$ una gramática con:\\ \\
                
                $N = \{$$<\text{number}>, <\text{firstdigit}>, <\text{restnumber}>, <\text{digit}>,$\\
                        $<\text{code}>,<\text{sentence}>,<\text{assignment}>,<\text{loop}>,<\text{id}>\}$\\
                $T = $  $ \Sigma_d \cup \Sigma_w$\\
                $P = \{$\\
            % \end{tabular}\\

            \begin{tabular}{ l l l l l }
                      &    &$<\text{number}>    $ & $\rightarrow$ & $<\text{firstdigit}>|<\text{firstdigit}> <\text{restnumber}>,$\\
                      &    &$<\text{firstdigit}>$ & $\rightarrow$ & $1| 2| 3| 4| 5| 6| 7| 8| 9,$\\
                      &    &$<\text{digit}>     $ & $\rightarrow$ & $0| 1| 2| 3| 4| 5| 6| 7| 8| 9,$\\
                      &    &$<\text{restnumber}>$ & $\rightarrow$ & $<\text{digit}>|<\text{digit}><\text{restnumber}>,$\\
                      &    &$<\text{code}>      $ & $\rightarrow$ & $<\text{sentence}>|<\text{code}>;<\text{sentence}>,$\\
                      &    &$<\text{sentence}>  $ & $\rightarrow$ & $<\text{assignment}>|<\text{loop}>,$\\
                      &    &$<\text{id}>        $ & $\rightarrow$ & $\X{} <\text{number}>,$\\
                      &    &$<\text{loop}>      $ & $\rightarrow$ & $\while <\text{id}>  \not = 0 \; \dokw <\text{code}> \od,$
            \end{tabular}\\
            \begin{tabular}{ l l l l l l }
                      &    &$<\text{assignment}>$ & $\rightarrow$ & $<\text{id}> \coloneqq 0              $ & $|<\text{id}> \coloneqq <\text{id}>|$\\
                      &    &                      &               & $<\text{id}> \coloneqq <\text{id}> + 1$ & $|<\text{id}> \coloneqq <\text{id}> - 1\}$\\
            \end{tabular}
            
            $\}$
        
        \end{solution}

        \part Para un programa $P$, con entrada $X1=n$ que calcula la función $H(n)$ y tiene complejidad
$T(n)$, dar una relación entre $H(n)$ y $T(n)$.
        \begin{solution}
            \\
            Dado un programa $P$ el cual calcula la función $H(n)$, con entrada $X1$, tiene complejidad $T(n)$, es decir, interpretando $P$ como un algoritmo, $O(P)=T(n)$. Debido a la ambigüedad del enunciado, nos vemos obligados a considerar dos casos:
            \begin{enumerate}
                \item El programa $P$ calcula única y exclusivamente la función $H(n)$, es decir, escribiendo dicho prorgama en EXWHILE:
                \\
                \\
                P = (H(n))
                \\
                \\
                Ahora bien, como $P = H(n)$, entonces $O(P) = O(H(n))$, por lo que $O(P) = T(n)$ y la complejidad temporal de P es $T(n)$. 
                \\
                \\
                Veamos esto anterior para un ejemplo, en el que la función $H(n) = n!$, es decir, $P$ es el programa que calcular el factorial de un número. Se puede comprobar que $T(n) = n$, o lo que es lo mismo, $O(P) = n$. 
                \\
                \\
                Esto es un claro ejemplo de que si estamos trabajando en este programa y tenemos una complejidad temporal superior; es decir, nuestro programa hace algunas iteraciones más, podemos concluir que hemos cometido algún error.

                \item El programa $P$ calcula en su interior la función $H(n)$, pero este puede realizar algunas operaciones más.
                \\
                \\
                En este caso, $P \neq H(n)$, pero $O(P) = T(n)$, por lo que $O(P) = T(n)$ por el propio enunciado. Solo que ahora, podría suceder que $P$ tuviera bucles anidados que hicieran que la complejidad de $P$ aumentara. Por ejemplo, sea $P$ un programa que calcula $(n!)^n$, y $H(n) = n!$, la misma función del ejercicio anterior. Es claro que $P$ calcula $H(n)$ en su interior. 
                \\
                \\
                Sin embargo, se puede demostrar que la complejidad de $P$ es $O(nlog(n))$, por tanto, la relación que se obtiene es:
                \begin{equation*}
                    O(P) = T(n) \geq O(H(n))  
                \end{equation*}
            \end{enumerate} 




        \end{solution}
    \end{parts}

    \question {\bfseries Ejercicio en la olimpiada matemática 2022
    \begin{enumerate}
        \item Es la hora de comer en el zoo y los animales se ponen en cola a la espera
        de que abran el comedor. Pero hoy están tardando en abrir y el hambre
        aprieta. Los animales se han situado en una improvisada cola, cumpliendo
        el distanciamiento de metro y medio y nadie los ha ordenado para evitar
        que, como consecuencia del hambre, se coman unos a otros.
        \item En este zoo, un animal se puede comer a otro de un tamaño hasta D
        centímetros (cm) inferior. Así, si D = 2, un animal que mida 14 cm puede
        comerse a otro de 13 cm o 12 cm, pero no a animales de 11 cm o menos.
        No podría comerse a animales de su mismo tamaño o superior.
        \item Cuando un animal se come a otro, ocupa la posición en la cola del animal
        que ha sido devorado y el resto de la cola se compacta volviendo todos al
        distanciamiento de metro y medio.
        \item La naturaleza es sabia pero cuando tiene hambre pierde las formas, así que
        los animales se comerán unos a otros en el orden en el que finalmente
        queden menos animales vivos.
        \item  Queremos saber cuántos animales sobrevivirán después de esta larga
        espera. Pensad en una solución por fuerza bruta. No hay que implementar
        la solución, bastará con describirla. ¿Cuánto tiempo se tardará en
        encontrar la solución si hay n animales?
    \end{enumerate}    
    }
    \begin{solution}
            \textbf{FUERZA BRUTA} 
            \\
            \\
            \\
            \\
            \\
            \\
            Existe una versión programada  y vamos a proceder a explicarla
            \\
            \\
            Partimos del mencionado array y del entero
            \\
            \(Animales[ ] = {7,3,5,4}\)
            \\
            \(Alt = 2\)

            \begin{enumerate}
                \item Recorremos todo el array
                \item Por cada posición vemos si esta puede comer o no:
                   \begin{enumerate}
                      \item Si puede comer, lo hace (se quita el comido del array) y se vuelve a llamar a esta función
                      \item Si no, no se hace nada
                   \end{enumerate}
                \item El algoritmo se queda, de todas las devueltas, con la cadena de menor tamaño
            \end{enumerate}
            $Complejidad = O(2^n)$
            \\
            \\
            \\
            \textbf{VORAZ} 



            También existe una versión programada de esta.
                    \begin{enumerate}
                \item Recorremos todo el array
                \item Por cada posición vemos cuántos vecinos se puede comer cada uno.
                \item El que pueda comer más será el elegido y se comerá a sus vecinos.
                \item Repetimos hasta que nadie pueda comer más
            \end{enumerate}
            $Complejidad = O(n^2)$
            
    \end{solution}
    

\end{questions}



%\hrule
%\subsection*{For retting}
%Ikke skriv noe her. \par \noindent
%\gradetable[h][questions]	
\end{document}