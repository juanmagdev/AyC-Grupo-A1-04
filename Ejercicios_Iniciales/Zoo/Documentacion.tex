% 	TEMPLATE DE RELACIÓN DE EJERCICIOS
%
% Creador: Juamagdev

% ------------------------------------------------------------------
% ---------------------- Package imports ---------------------------
% ------------------------------------------------------------------
\documentclass[11pt, a4paper]{exam}
\usepackage[utf8]{inputenc}		% UTF-8
\usepackage[spanish]{babel}		% Idioma
\usepackage{graphicx}			% Para importar gráficos
\usepackage{amsthm, amsfonts, amssymb, amssymb} % Todos los paquetes AMS
\usepackage{mathtools}          % Arregla bugs de AMS
\usepackage{hyperref}			% Para \href{URL}{text}
\usepackage{enumitem}			% Para enumerar
\usepackage{color} 				% Para definir colores nuevos
\usepackage{lastpage}			% Para \pageref{LastPage}
\usepackage{booktabs}					% Tablas profesionales
\usepackage{parskip}					% Espacio de párrafos
\usepackage[sharp]{easylist}			% Para litas
\usepackage[expansion=false]{microtype} % Soluciona bugs de tipografias
\usepackage[margin=2.25cm, includehead, includefoot]{geometry}
\usepackage{cancel}
\usepackage{pgfplots}
\usepackage{listings}



% ------------------------------------------------------------------
% ---------------------- Constantes --------------------------------
% ------------------------------------------------------------------
% Las constantes mas importantes
\newcommand{\mytitle}{Ejercicio Inicial 5: Zoo}
\newcommand{\mysubject}{Algoritmia y Complejidad}
\newcommand{\mydate}{\today}
\newcommand{\myauthor}{Salvador Durán Borrego}

% Redefinir para cada caso
\newcommand{\myrhead}{Universidad de Málaga}
\newcommand{\mypagename}{Página}
\newcommand{\mycreated}{Creado por: }
\newcommand{\mysolname}{Solución}
\renewcommand{\solutiontitle}{\noindent\textbf{\mysolname.}\hspace{0.75em}}
\pointpoints{point}{points}

\providecommand{\abs}[1]{\lvert#1\rvert}

% ------------------------------------------------------------------
% ---------------------- Ajustes -----------------------------------
% ------------------------------------------------------------------
% \shadedsolutions
\printanswers % Alternative: \noprintanswers, \printanswers
%\rhead{{\scshape {\footnotesize  \myrhead}}}
%\cfoot{\mypagename \enspace \thepage}
\definecolor{SolutionColor}{rgb}{0.8,0.9,1} % light blue

% Cabeceras y pie de página
\runningfootrule
\firstpagefootrule
\firstpagefooter{\mysubject}{}{\mypagename\ \thepage\ de \pageref*{LastPage}}
\runningfooter{\mysubject}{}{\mypagename\ \thepage\ de \pageref*{LastPage}}
\runningheader{}{}{\includegraphics[width = 3 cm]{figs/logo.pdf}}
\firstpageheadrule
\firstpageheader{\mydate}{\mytitle}{\pageref*{LastPage} páginas en total}

% Answer command for double lines
\def\answer#1{\underline{\underline{#1}}}

% ------------------------------------------------------------------
% ---------------------- Documento ---------------------------------
% ------------------------------------------------------------------
\begin{document}
\pagestyle{headandfoot}
\noindent {\scshape \Large  \mytitle
    \ifprintanswers
        \enspace (\mysolname	 )
    \fi
} \\
\noindent {\mycreated \enspace  \myauthor} \vspace{1em}
\hrule \hrule
\vspace{5mm}

% ------------------------------------------------------------------
% ---------------------- Contenido ---------------------------------
% ------------------------------------------------------------------

\begin{questions}
    \addpoints
    \question {\bfseries Dado un array de enteros que representa la altura y posición de unos animales y un entero que representa la diferencia de altura máxima que puede haber para que un animal se coma a otro contiguo (la mínima es 1), crea una solución: }
    
    
    \begin{parts}
        \part Fuerza Bruta
        \begin{solution}
            \\
            
            Existe una versión programada  y vamos a proceder a explicarla
            \\
            \\
            Partimos del mencionado array y del entero
            \\
            \(Animales[ ] = {7,3,5,4}\)
            \\
            \(Alt = 2\)

            \begin{enumerate}
                \item Recorremos todo el array
                \item Por cada posición vemos si esta puede comer o no:
                   \begin{enumerate}
                      \item Si puede comer, lo hace (se quita el comido del array) y se vuelve a llamar a esta función
                      \item Si no, no se hace nada
                   \end{enumerate}
                \item El algoritmo se queda, de todas las devueltas, con la cadena de menor tamaño
            \end{enumerate}
            $Complejidad = O()$
        
        \end{solution}

        \part Voraz

        \begin{solution}
        También existe una versión programada de esta.
                    \begin{enumerate}
                \item Recorremos todo el array
                \item Por cada posición vemos cuántos vecinos se puede comer cada uno.
                \item El que pueda comer más será el elegido y se comerá a sus vecinos.
                \item Repetimos hasta que nadie pueda comer más
            \end{enumerate}
            $Complejidad = O(n^2)$



        
        \end{solution}
    \end{parts}

\end{questions}



%\hrule
%\subsection*{For retting}
%Ikke skriv noe her. \par \noindent
%\gradetable[h][questions]	
\end{document}